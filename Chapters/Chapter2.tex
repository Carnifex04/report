% Chapter 2
\chapter{Literature Review}
\label{Chapter2}

\section{Introduction}
Algorithmic trading has gained popularity as an approach to financial trading, leveraging advanced mathematical and statistical models to identify profitable trading opportunities and execute trades without human intervention. The literature has demonstrated the effectiveness of algorithmic trading strategies in generating consistent profits in the financial markets, outperforming traditional buy-and-hold strategies and manual trading approaches.

\section{Early Works}
Lo and MacKinlay (1990) proposed a statistical method for identifying and exploiting short-term trading opportunities in the stock market using a moving average crossover strategy. Their approach showed superior performance compared to a buy-and-hold strategy. Later, various technical indicators and trading strategies, such as Bollinger bands, MACD, RSI, and momentum strategies, were proposed.

\section{Strategy Development}
Algorithmic trading involves using computer programs to automatically execute trades based on predetermined rules and strategies. The literature review highlights the different techniques used in strategy development, including machine learning techniques such as neural networks and support vector machines for predicting stock prices and developing trading strategies.

\section{Backtesting}
Backtesting is a crucial aspect of algorithmic trading, involving testing the effectiveness of trading strategies using historical data. Accurate and realistic backtesting is essential to avoid the problem of overfitting, which occurs when a trading strategy performs well on historical data but fails to generalize to new data.

\section{Optimization}
Optimization involves finding the optimal parameters for a given trading strategy to maximize its performance. The literature review explores different methods for optimizing trading strategies, including genetic algorithms, simulated annealing, and particle swarm optimization.

\section{Execution}
Execution is a crucial aspect of algorithmic trading, involving the actual implementation of trades based on trading strategies. Several studies have emphasized the importance of minimizing transaction costs and market impact while executing trades.

\section{Conclusion}
In conclusion, the literature review emphasizes the vast body of research on algorithmic trading and explores its different aspects, including strategy development, backtesting, optimization, and execution. Accurate and realistic backtesting, optimization, and execution are essential for successful algorithmic trading.